% =========================================================
%  1. 核心宏包与环境配置 (Core Packages)
% =========================================================
% 确保 tcolorbox 库已加载 (ElegantBook 已加载基础包,这里加载扩展库)
\tcbuselibrary{skins, breakable, theorems}

% 修复数学公式编号格式 (防止报错)
\renewcommand{\theequation}{\thechapter.\arabic{equation}}

% =========================================================
%  2. 表格与列格式定义 (Fix: Illegal pream-token L)
% =========================================================
% 定义左对齐自适应列 'L',解决 3.tex 中的表格报错
% 必须在 main.tex 加载了 tabularx 和 array 包后生效
\newcolumntype{L}{>{\raggedright\arraybackslash}X}

% 表格单元格颜色辅助命令
\newcommand{\ccr}[1]{\makecell{{\color{#1}\rule{1cm}{1cm}}}}

% =========================================================
%  3. 莫兰迪配色方案 (Morandi Style)
% =========================================================
% 定义低饱和度护眼色
\definecolor{MorandiBlue}{RGB}{102, 129, 149}     % 雾霾蓝 (主色)
\definecolor{MorandiGreen}{RGB}{138, 157, 122}    % 橄榄绿 (定义/例题)
\definecolor{MorandiYellow}{RGB}{204, 169, 114}   % 姜黄 (定理/注意)
\definecolor{MorandiRed}{RGB}{196, 142, 142}      % 豆沙红 (结论/警告)
\definecolor{MorandiBrown}{RGB}{166, 153, 140}    % 暖灰 (引用)

% 强制覆盖 ElegantBook 全局变量,实现换肤
\definecolor{structurecolor}{named}{MorandiBlue} 
\definecolor{main}{named}{MorandiGreen}
\definecolor{second}{named}{MorandiYellow}
\definecolor{third}{named}{MorandiRed}
\colorlet{coverlinecolor}{MorandiBlue}

% =========================================================
%  4. 自定义文本框定义 (Fix: Environment undefined)
% =========================================================

% 1. 核心思想框 (Core Box) - 蓝色系
\newtcolorbox{corebox}[1]{
  enhanced,
  colback=structurecolor!10!white,
  colframe=structurecolor,
  fonttitle=\bfseries,
  title={#1},
  attach boxed title to top left={yshift=-2mm, xshift=2mm},
  boxed title style={colback=structurecolor},
  breakable
}

% 2. 例题框 (Example Box) - 绿色系
\newtcolorbox{examplebox}[1]{
  enhanced,
  colback=main!10!white,
  colframe=main,
  fonttitle=\bfseries,
  title={#1},
  attach boxed title to top left={yshift=-2mm, xshift=2mm},
  boxed title style={colback=main},
  breakable
}

% 3. 警告/陷阱框 (Alert Box) - 红色系
\newtcolorbox{alertbox}[1]{
  enhanced,
  colback=third!10!white,
  colframe=third,
  fonttitle=\bfseries,
  title={#1},
  attach boxed title to top left={yshift=-2mm, xshift=2mm},
  boxed title style={colback=third},
  breakable
}

% 4. 教材引用框 (Book Box) - 棕色系
\newtcolorbox{bookbox}[1]{
  enhanced,
  colback=MorandiBrown!10!white,
  colframe=MorandiBrown,
  fonttitle=\bfseries,
  title={#1},
  attach boxed title to top left={yshift=-2mm, xshift=2mm},
  boxed title style={colback=MorandiBrown},
  breakable
}

% =========================================================
%  5. 信号与系统专用简写 (Fix: Undefined control sequence)
% =========================================================
% 使用 providecommand 确保不冲突,然后 renew 强制生效
\providecommand{\ak}{} \renewcommand{\ak}{a_k}       % 傅里叶级数系数
\providecommand{\wo}{} \renewcommand{\wo}{\omega_0}  % 基波角频率
\providecommand{\jw}{} \renewcommand{\jw}{j\omega}   % 虚频
\providecommand{\dd}{} \renewcommand{\dd}{\,\mathrm{d}} % 微分算子
\providecommand{\FT}{} \renewcommand{\FT}{\longleftrightarrow} % 傅里叶变换箭头

% =========================================================
%  6. 通用数学符号与工具
% =========================================================
\usepackage{listofitems}
\usepackage{dsfont}

% 循环命令
\newcommand\cycle[2][\,]{%
  \readlist\thecycle{#2}%
  (\foreachitem\i\in\thecycle{\ifnum\icnt=1\else#1\fi\i})%
}

% 常用算子
\providecommand{\iff}{} \renewcommand{\iff}{\Leftrightarrow}
\newcommand{\z}{\left}
\newcommand{\y}{\right}
\newcommand{\sss}{\sum_{i=1}^{n}}
\newcommand{\ve}{\varepsilon}

% 数集符号 (\R, \N 等)
\newcommand{\N}{\mathbb{N}}
\newcommand{\Z}{\mathbb{Z}}
\newcommand{\Q}{\mathbb{Q}}
\newcommand{\R}{\mathbb{R}}
\newcommand{\C}{\mathbb{C}}
\newcommand{\FF}{\mathbb{F}}

% 高级数学算子定义
\DeclareMathOperator{\diam}{diam}
\DeclareMathOperator{\ima}{im}
\DeclareMathOperator{\spn}{span}
\DeclareMathOperator{\Hom}{Hom}
\DeclareMathOperator{\Tr}{Tr}
\renewcommand{\O}{\mathcal{O}}
\renewcommand{\P}{\mathcal{P}}
\DeclareMathOperator{\del}{\partial}