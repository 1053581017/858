\chapter{离散时间信号的傅里叶变换}

% ==========================================
% 第一部分:离散傅里叶级数 (DFS)
% ==========================================
\section{离散傅里叶级数 (Discrete Fourier Series, DFS)}

\begin{introduction}[基本概念]
    任何一个周期的离散信号,都可以分解成有限个复指数信号的加权和。
\end{introduction}

\subsection{1. 周期信号的定义}
如果一个离散信号 $f[n]$ 满足 $f[n+N] = f[n]$,那么它是周期的,周期为 $N$。

\subsection{2. 基波与谐波}
\begin{itemize}
    \item \textbf{基波频率}:定义 $\Omega_0 = \frac{2\pi}{N}$。这代表信号中最低的频率成分。
    \item \textbf{复指数信号}:$e^{jk\Omega_0 n}$。$k$ 代表第几次谐波(频率是基波的 $k$ 倍)。
\end{itemize}

\subsection{3. 离散时间特有的性质(非常重要!)}
离散信号与连续信号的一个巨大区别在于\textbf{频率的周期性}。公式证明了:
\[ e^{j(k+N)\Omega_0 n} = e^{jk\Omega_0 n} \]
\textbf{这意味着什么?}
这意味着频率 $k$ 和频率 $k+N$ 是完全一样的!
\begin{itemize}
    \item 在\textbf{连续时间}里,傅里叶级数需要从 $-\infty$ 加到 $+\infty$,因为高频谐波是无限的。
    \item 在\textbf{离散时间}里,因为 $k$ 和 $k+N$ 重复,我们只需要 \textbf{$N$ 个} 不同的复指数信号(通常取 $k=0$ 到 $N-1$)就足够描述整个信号了。
\end{itemize}

\subsection{4. DFS 核心公式}

\begin{corebox}{合成公式 (Synthesis Equation / 逆变换)}
    \[ f[n] = \sum_{k=0}^{N-1} a_k e^{jk\Omega_0 n} \]
    \textbf{物理意义}:时域信号 $f[n]$ 是由 $N$ 个不同频率的复指数波叠加而成的。
    \\ \textbf{$a_k$}:称为傅里叶系数(或频谱系数),它代表了第 $k$ 个频率成分在原信号中占了多大比重(幅度和相位)。
\end{corebox}

\begin{corebox}{分析公式 (Analysis Equation)}
    \[ a_k = \frac{1}{N} \sum_{n=0}^{N-1} f[n] e^{-jk\Omega_0 n} \]
    \textbf{物理意义}:这是将时域信号 $f[n]$ 转换到频域。计算出的 $a_k$ 告诉我们在频率 $k\Omega_0$ 处信号的强度。
\end{corebox}

\begin{note}
    \textbf{求和范围}:求和区间不一定要是 $0$ 到 $N-1$,只要是任意连续的 $N$ 个点(记作 $n=<N>$)都可以,因为信号是周期的。
\end{note}

\subsection{5. 总结:DFS 对偶性 (Duality)}
看完这两页,你需要记住的一组核心关系是:
\begin{itemize}
    \item \textbf{时域 $f[n]$}:是周期的,周期为 $N$。
    \item \textbf{频域 $a_k$}:也是周期的,周期为 $N$(因为 $e^{-j(k+N)\Omega_0 n} = e^{-jk\Omega_0 n}$)。
\end{itemize}

\newpage

% ==========================================
% 第二部分:离散时间周期信号的 DTFT
% ==========================================
\section{离散时间周期信号的 DTFT}

本节推导离散时间周期信号(Discrete-Time Periodic Signal)的离散时间傅里叶变换(DTFT)公式。

\subsection{1. 基础定义}
假设 $x[n]$ 是一个周期为 $N$ 的离散时间信号,即满足 $x[n] = x[n+N]$。根据离散傅里叶级数(DFS),我们可以将这个周期信号分解为复指数信号的加权和:
\[ x[n] = \sum_{k=0}^{N-1} a_k e^{j \frac{2\pi}{N} k n} \]
其中:
\begin{itemize}
    \item $N$ 是信号的周期。
    \item $a_k$ 是傅里叶级数系数(DFS系数)。
    \item $\frac{2\pi}{N}$ 是基波频率。
\end{itemize}
\small{注意:系数 $a_k$ 本身也是周期的,即 $a_k = a_{k+N}$。因此,虽然求和通常写为 $0$ 到 $N-1$,但我们在概念上可以认为 $k$ 覆盖所有整数,且 $a_k$ 会重复。}

\subsection{2. 推导过程}

\paragraph{Step 1: 应用 DTFT 定义}
离散时间傅里叶变换(DTFT)的定义式为:
\[ X(e^{j\omega}) = \sum_{n=-\infty}^{\infty} x[n] e^{-j\omega n} \]
我们将 $x[n]$ 的 DFS 表达式代入 DTFT 定义式:
\[ X(e^{j\omega}) = \sum_{n=-\infty}^{\infty} \left( \sum_{k=0}^{N-1} a_k e^{j \frac{2\pi}{N} k n} \right) e^{-j\omega n} \]

\paragraph{Step 2: 交换求和顺序}
根据线性性质,我们可以交换求和顺序,将 $a_k$ 提到前面:
\[ X(e^{j\omega}) = \sum_{k=0}^{N-1} a_k \left[ \sum_{n=-\infty}^{\infty} e^{j \frac{2\pi}{N} k n} \cdot e^{-j\omega n} \right] \]
方括号内的部分就是复指数信号 $e^{j \omega_0 n}$ 的 DTFT,其中 $\omega_0 = \frac{2\pi}{N}k$。

\paragraph{Step 3: 利用复指数信号的 DTFT 变换对}
对于离散时间复指数信号 $e^{j\omega_0 n}$,其 DTFT 是在频率 $\omega_0$ 及其 $2\pi$ 周期延拓处的冲激函数串:
\[ \text{DTFT}\{ e^{j\omega_0 n} \} = \sum_{r=-\infty}^{\infty} 2\pi \delta(\omega - \omega_0 - 2\pi r) \]
将 $\omega_0 = \frac{2\pi}{N}k$ 代入上式,方括号内的部分变为:
\[ \sum_{r=-\infty}^{\infty} 2\pi \delta\left(\omega - \frac{2\pi}{N}k - 2\pi r\right) \]

\paragraph{Step 4: 整合与化简}
将 Step 3 的结果代回方程中:
\[ X(e^{j\omega}) = \sum_{k=0}^{N-1} a_k \left[ \sum_{r=-\infty}^{\infty} 2\pi \delta\left(\omega - \left(\frac{2\pi}{N}k + 2\pi r\right)\right) \right] \]
\textbf{关键变换}:这是一个双重求和。
\begin{itemize}
    \item $k$ 在 $[0, N-1]$ 内变化。
    \item $r$ 在 $(-\infty, \infty)$ 内变化。
\end{itemize}
观察 $\delta$ 函数中的频率位置 $\frac{2\pi}{N}(k + Nr)$。令 $m = k + Nr$,当 $k$ 和 $r$ 遍历各自范围时,$m$ 遍历所有整数。又因为 $a_k$ 是周期的 ($a_k = a_{k+Nr}$),我们可以合并双重求和为一个单一求和:

\subsection{3. 最终结论}
\begin{corebox}{周期信号的 DTFT 公式}
    \[ X(e^{j\omega}) = \sum_{k=-\infty}^{\infty} 2\pi a_k \delta\left(\omega - \frac{2\pi}{N}k\right) \]
\end{corebox}
\textbf{物理意义}:
\begin{enumerate}
    \item \textbf{频谱离散}:只在基波频率 $\frac{2\pi}{N}$ 的整数倍处有值。
    \item \textbf{能量集中}:在这些频率点上,频谱表现为强度为 $2\pi a_k$ 的冲激函数(Dirac delta function)。
\end{enumerate}
5. 直流信号信号: $x[n] = 1$ (对所有 $n$)证明:这是一个周期性的功率信号,其变换包含冲激函数。我们知道 $\delta[n]$ 的反变换是 $1/2\pi \int_{-\pi}^{\pi} 1 \cdot e^{j\Omega n} d\Omega$。对于频域中的周期冲激串 $X(\Omega) = 2\pi \sum_{l=-\infty}^{\infty} \delta(\Omega - 2l\pi)$,我们求其 DTFT 反变换(IDTFT):$$x[n] = \frac{1}{2\pi} \int_{-\pi}^{\pi} [2\pi \sum_{l=-\infty}^{\infty} \delta(\Omega - 2l\pi)] e^{j\Omega n} d\Omega$$在 $(-\pi, \pi]$ 区间内,只有 $l=0$ 的项 $\delta(\Omega)$ 有效:$$x[n] = \frac{1}{2\pi} \int_{-\pi}^{\pi} 2\pi \delta(\Omega) e^{j\Omega n} d\Omega = \int_{-\pi}^{\pi} \delta(\Omega) \cdot 1 d\Omega = 1$$因为 IDTFT 得到 $1$,所以 $1$ 的 DTFT 是周期冲激串。$$X(\Omega) = 2\pi \sum_{l=-\infty}^{\infty} \delta(\Omega - 2l\pi)$$证毕。
斜坡加权指数信号: $(n+1)a^n u[n], \quad |a|<1$证明:利用时域卷积定理:$y[n] = x[n] * h[n] \leftrightarrow Y(\Omega) = X(\Omega)H(\Omega)$。观察信号形式,令 $x[n] = a^n u[n]$。计算 $x[n] * x[n]$:$$(a^n u[n]) * (a^n u[n]) = \sum_{k=-\infty}^{\infty} a^k u[k] a^{n-k} u[n-k]$$非零区间为 $k \ge 0$ 且 $n-k \ge 0$ (即 $k \le n$)。若 $n<0$,结果为0。对于 $n \ge 0$:$$\sum_{k=0}^{n} a^k a^{n-k} = \sum_{k=0}^{n} a^n = a^n \sum_{k=0}^{n} 1 = a^n (n+1)$$所以 $(n+1)a^n u[n]$ 正好是 $a^n u[n]$ 自卷积的结果。其频域变换即为 $a^n u[n]$ 变换的平方:$$X(\Omega) = \left( \frac{1}{1 - ae^{-j\Omega}} \right)^2$$证毕。12. 高阶加权指数信号: $\frac{(n+r-1)!}{n!(r-1)!} a^n u[n]$证明:这是序号 11 的推广。序号 1 的变换是 $H(\Omega) = (1-ae^{-j\Omega})^{-1}$。序号 11 的变换是 $H^2(\Omega)$,对应 2 个 $a^n u[n]$ 卷积。该通项公式实际上是 $r$ 个 $a^n u[n]$ 连续卷积的结果(组合数学中的重复组合数)。由卷积定理,频域为:$$X(\Omega) = [ \text{DTFT}\{a^n u[n]\} ]^r = \frac{1}{(1 - ae^{-j\Omega})^r}$$证毕。13. 周期冲激串信号: $\sum_{k=-\infty}^{\infty} \delta[n - kN]$证明:这是一个周期为 $N$ 的信号。根据泊松求和公式(Poisson Summation Formula),或者将其视为周期信号展开为离散傅里叶级数(DFS)。令 $x[n] = \sum_k \delta[n - kN]$。其傅里叶级数系数 $a_k$ 为:$$a_k = \frac{1}{N} \sum_{n=0}^{N-1} x[n] e^{-jk(2\pi/N)n} = \frac{1}{N} \sum_{n=0}^{N-1} \delta[n] e^{-jk(2\pi/N)n} = \frac{1}{N} \cdot 1 = \frac{1}{N}$$对于周期信号,其 DTFT 是在谐波频率 $\frac{2\pi k}{N}$ 处的冲激串,强度为 $2\pi a_k$。$$X(\Omega) = \sum_{k=-\infty}^{\infty} 2\pi a_k \delta(\Omega - \frac{2\pi k}{N}) = \frac{2\pi}{N} \sum_{k=-\infty}^{\infty} \delta(\Omega - \frac{2k\pi}{N})$$证毕。14. 一般周期信号信号: 周期为 $N$ 的周期信号证明:任何周期为 $N$ 的离散信号 $\tilde{x}[n]$ 都可以展开为傅里叶级数:$$\tilde{x}[n] = \sum_{k=0}^{N-1} a_k e^{j k \frac{2\pi}{N} n}$$其中 $a_k$ 是傅里叶系数。根据线性性质和复指数信号的变换(序号6):我们知道 $e^{j\Omega_0 n} \leftrightarrow 2\pi \sum_l \delta(\Omega - \Omega_0 - 2l\pi)$。这里 $\Omega_0 = k \frac{2\pi}{N}$。$$X(\Omega) = \sum_{k=0}^{N-1} a_k \left[ 2\pi \sum_{l=-\infty}^{\infty} \delta(\Omega - \frac{2\pi k}{N} - 2\pi l) \right]$$由于 $k$ 在 $0$ 到 $N-1$ 遍历,而 $l$ 在 $-\infty$ 到 $\infty$ 遍历,且 $\frac{2\pi k}{N} + 2\pi l = \frac{2\pi (k + lN)}{N}$。这实际上覆盖了所有 $\frac{2\pi}{N}$ 的整数倍。通常可以简写为求和覆盖所有 $k$(将 $l$ 合并进 $k$ 的范围,或者如表中所示保持 $l$ 的周期性结构,表中公式实际上省略了 $l$ 的求和符号,意指在整个频域轴上重复):表中的写法是 $2\pi \sum_{k=-\infty}^{\infty} a_k \delta(\Omega - \frac{2k\pi}{N})$。注意:这里的 $a_k$ 是周期为 $N$ 的系数序列(即 $a_{k+N} = a_k$),所以这个求和式正确地表示了所有谐波分量的冲激函数。证毕。