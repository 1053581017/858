\chapter{离散时间信号的傅里叶变换}
离散傅里叶级数 (Discrete Fourier Series, DFS)
第一部分:基本概念与合成公式(任何一个周期的离散信号,都可以分解成有限个复指数信号的加权和。)
1. 周期信号的定义如果一个离散信号 $f[n]$ 满足 $f[n+N] = f[n]$,那么它是周期的,周期为 $N$。2. 基波与谐波基波频率:定义 $\Omega_0 = \frac{2\pi}{N}$。这代表信号中最低的频率成分。复指数信号:$e^{jk\Omega_0 n}$。$k$ 代表第几次谐波(频率是基波的 $k$ 倍)。
3. 离散时间特有的性质(非常重要!)文中特别指出了离散信号与连续信号的一个巨大区别:频率的周期性。公式 (6.2.1) 证明了:$$e^{j(k+N)\Omega_0 n} = e^{jk\Omega_0 n}$$这意味着什么?这意味着频率 $k$ 和频率 $k+N$ 是完全一样的!在连续时间里,傅里叶级数需要从 $-\infty$ 加到 $+\infty$,因为高频谐波是无限的。在离散时间里,因为 $k$ 和 $k+N$ 重复,我们只需要 $N$ 个 不同的复指数信号(通常取 $k=0$ 到 $N-1$)就足够描述整个信号了。4. 合成公式 (Synthesis Equation)这就是公式 (6.2.2),也叫逆变换:$$f[n] = \sum_{k=0}^{N-1} a_k e^{jk\Omega_0 n}$$
物理意义:时域信号 $f[n]$ 是由 $N$ 个不同频率的复指数波叠加而成的。$a_k$:称为傅里叶系数(或频谱系数),它代表了第 $k$ 个频率成分在原信号中占了多大比重(幅度和相位)。
分析公式 (Analysis Equation)这就是公式 (6.2.5) 和 (6.2.7):$$a_k = \frac{1}{N} \sum_{n=0}^{N-1} f[n] e^{-jk\Omega_0 n}$$(注:原文中公式6.2.4是用 $m$ 表示的,6.2.5换回了习惯用的 $k$,本质一样。)物理意义:这是将时域信号 $f[n]$ 转换到频域。计算出的 $a_k$ 告诉我们在频率 $k\Omega_0$ 处信号的强度。求和范围:公式 (6.2.6) 和 (6.2.7) 补充说明,求和区间不一定要是 $0$ 到 $N-1$,只要是任意连续的 $N$ 个点(记作 $n=<N>$)都可以,因为信号是周期的。
总结:DFS 对偶性 (Duality)看完这两页,你需要记住的一组核心关系是:时域 $f[n]$:是周期的,周期为 $N$。频域 $a_k$:也是周期的,周期为 $N$(因为 $e^{-j(k+N)\Omega_0 n} = e^{-jk\Omega_0 n}$)。

离散时间周期信号(Discrete-Time Periodic Signal)的离散时间傅里叶变换(DTFT)**公式。
1. 基础定义假设 $x[n]$ 是一个周期为 $N$ 的离散时间信号,即满足 $x[n] = x[n+N]$。根据离散傅里叶级数(DFS),我们可以将这个周期信号分解为复指数信号的加权和:$$x[n] = \sum_{k=0}^{N-1} a_k e^{j \frac{2\pi}{N} k n}$$其中:$N$ 是信号的周期。$a_k$ 是傅里叶级数系数(DFS系数)。$\frac{2\pi}{N}$ 是基波频率。注意:系数 $a_k$ 本身也是周期的,即 $a_k = a_{k+N}$。因此,虽然求和通常写为 $0$ 到 $N-1$,但我们在概念上可以认为 $k$ 覆盖所有整数,且 $a_k$ 会重复。2. 应用 DTFT 定义离散时间傅里叶变换(DTFT)的定义式为:$$X(e^{j\omega}) = \sum_{n=-\infty}^{\infty} x[n] e^{-j\omega n}$$我们将步骤 1 中的 $x[n]$ 表达式代入 DTFT 定义式:$$X(e^{j\omega}) = \sum_{n=-\infty}^{\infty} \left( \sum_{k=0}^{N-1} a_k e^{j \frac{2\pi}{N} k n} \right) e^{-j\omega n}$$3. 交换求和顺序根据线性性质,我们可以交换求和顺序,将 $a_k$ 提到前面:$$X(e^{j\omega}) = \sum_{k=0}^{N-1} a_k \left[ \sum_{n=-\infty}^{\infty} e^{j \frac{2\pi}{N} k n} \cdot e^{-j\omega n} \right]$$方括号内的部分就是复指数信号 $e^{j \omega_0 n}$ 的 DTFT,其中 $\omega_0 = \frac{2\pi}{N}k$。4. 复指数信号的 DTFT我们需要用到一个关键的变换对。对于离散时间复指数信号 $e^{j\omega_0 n}$,其 DTFT 是在频率 $\omega_0$ 及其 $2\pi$ 周期延拓处的冲激函数串:$$\text{DTFT}\{ e^{j\omega_0 n} \} = \sum_{r=-\infty}^{\infty} 2\pi \delta(\omega - \omega_0 - 2\pi r)$$将 $\omega_0 = \frac{2\pi}{N}k$ 代入上式,方括号内的部分变为:$$\sum_{r=-\infty}^{\infty} 2\pi \delta\left(\omega - \frac{2\pi}{N}k - 2\pi r\right)$$5. 整合与化简将步骤 4 的结果代回步骤 3 的方程中:$$X(e^{j\omega}) = \sum_{k=0}^{N-1} a_k \left[ \sum_{r=-\infty}^{\infty} 2\pi \delta\left(\omega - \left(\frac{2\pi}{N}k + 2\pi r\right)\right) \right]$$这里我们有一个双重求和:$k$ 从 $0$ 到 $N-1$(一个周期内的频率分量)。$r$ 从 $-\infty$ 到 $\infty$(频谱的 $2\pi$ 周期性复制)。我们可以观察 $\delta$ 函数中的频率位置:$\frac{2\pi}{N}(k + Nr)$。因为 $k$ 是在一个周期 $N$ 内变化的,而 $r$ 是任意整数,所以表达式 $k + Nr$ 实际上可以表示为任意整数。让我们定义一个新的整数索引 $m = k + Nr$。当 $k$ 遍历 $0$ 到 $N-1$ 且 $r$ 遍历 $-\infty$ 到 $\infty$ 时,$m$ 将遍历从 $-\infty$ 到 $\infty$ 的所有整数。同时,由于 DFS 系数 $a_k$ 是周期的(周期为 $N$),即 $a_k = a_{k+Nr} = a_m$。因此,我们可以将双重求和合并为一个从 $-\infty$ 到 $\infty$ 的单一求和,并将索引变量重新命名为 $k$(对应图片中的公式):$$X(e^{j\omega}) = \sum_{k=-\infty}^{\infty} 2\pi a_k \delta\left(\omega - \frac{2\pi}{N}k\right)$$结论这就是图片中公式的由来。它物理意义是:周期信号的频谱是离散的:只在基波频率 $\frac{2\pi}{N}$ 的整数倍处有值。能量集中在冲激上:在这些频率点上,频谱表现为强度为 $2\pi a_k$ 的冲激函数(Dirac delta function)。