\chapter{滤波、调制和采样}

% ==========================================
% 第一部分:核心概念解析
% ==========================================
\section{核心概念解析:基带信号 vs. 频带信号}

在信号处理和通信中,通常根据频谱在频率轴上的位置,将信号分为两大类:

\subsection{1. 基带信号 (Baseband Signal)}
\begin{itemize}
    \item \textbf{定义}:频谱分量集中在\textbf{零频率(DC)}附近的信号。
    \item \textbf{特征}:频带从 $0$ (或非常接近 $0$) 开始,延伸到某个最高频率 $f_{max}$。
    \item \textbf{直观理解}:这是未经调制的原始信号。
    \item \textbf{例如}:
    \begin{itemize}
        \item 你对着麦克风说话的声音(声波转换成的电信号)。
        \item 上一题中的图像 (b),频谱横跨 $[-W_2, W_2]$,中心就在 $0$。
    \end{itemize}
    \item \textbf{对应采样定理}:\textbf{奈奎斯特采样定理 (Nyquist Sampling Theorem)}。
    \begin{itemize}
        \item \textbf{公式}:$f_s \ge 2 f_{max}$。
    \end{itemize}
\end{itemize}

\subsection{2. 频带信号 (Passband / Bandpass Signal)}
\begin{itemize}
    \item \textbf{定义}:频谱被搬移到较高频率处,且\textbf{不包含}零频率分量的信号。
    \item \textbf{特征}:频带位于 $f_{min}$ 到 $f_{max}$ 之间,且 $f_{min} \gg 0$。
    \begin{itemize}
        \item \textbf{带宽}:$B = f_{max} - f_{min}$。
    \end{itemize}
    \item \textbf{直观理解}:这是经过调制后的信号,适合无线传输。
    \item \textbf{例如}:
    \begin{itemize}
        \item 收音机接收的 FM/AM 广播信号。
        \item Wi-Fi 信号。
    \end{itemize}
    \item \textbf{对应采样定理}:\textbf{带通采样定理 (Bandpass Sampling Theorem)}。
    \begin{itemize}
        \item \textbf{公式}:$f_s = \frac{2f_{max}}{m}$(其中 $m$ 是整数)。
        \item \textit{注:这个公式仅适用于此类信号,不可混用。}
    \end{itemize}
\end{itemize}

% ==========================================
% 第二部分:带通采样定理
% ==========================================
\section{带通信号的采样定理(欠采样)}

\begin{corebox}{核心问题}
    对于一个频率范围为 $[f_L, f_H]$,带宽为 $B = f_H - f_L$ 的带通信号,为了避免混叠,采样频率 $f_s$ 必须使得信号频谱完整地“停”在第 $m$ 个奈奎斯特区间内。
\end{corebox}

\subsection{推导:停车位模型}
我们将频谱比喻为一辆车,奈奎斯特区间比喻为停车位。第 $m$ 个区间的范围为:
\[
\text{下界} = (m-1)\frac{f_s}{2} \quad \text{至} \quad \text{上界} = m\frac{f_s}{2}
\]
为了让信号 $[f_L, f_H]$ (即整个“车”)停进这个“车位”里,必须同时满足两个条件:

\begin{enumerate}
    \item \textbf{条件 1:信号没超出上界(车头没越线)}
    \[ f_H \le m\frac{f_s}{2} \implies \boldsymbol{f_s \ge \frac{2f_H}{m}} \]
    这就是公式的下限。
    
    \item \textbf{条件 2:信号没超出下界(车尾没越线)}
    \[ f_L \ge (m-1)\frac{f_s}{2} \implies \boldsymbol{f_s \le \frac{2f_L}{m-1}} \]
    这就是公式的上限。
\end{enumerate}

\subsection{核心公式汇总}

\begin{alertbox}{带通采样核心不等式}
    合并上述两个条件,得到采样频率 $f_s$ 需满足:
    \[ \boxed{ \frac{2f_H}{m} \le f_s \le \frac{2f_L}{m-1} } \]
    其中,$m$ 是一个正整数,为了保证不等式有解(上限 $\ge$ 下限),其最大取值受限于带宽 $B$:
    \[ m \le m_{max} = \left\lfloor \frac{f_H}{B} \right\rfloor \]
    (注:$\lfloor \cdot \rfloor$ 表示向下取整)
\end{alertbox}

\subsection{详细计算步骤}

\begin{examplebox}{解题算法流程}
\begin{enumerate}
    \item \textbf{确定参数}:找出信号的最高频率 $f_H$ 和最低频率 $f_L$,计算带宽 $B = f_H - f_L$。
    
    \item \textbf{计算 $m$ 的最大值}:
    \[ m_{max} = \left\lfloor \frac{f_H}{B} \right\rfloor \]
    这意味着你可以把频谱“折叠”多少次而不发生重叠。
    
    \item \textbf{寻找可行的采样区间}:
    对于 $m = 1, 2, \dots, m_{max}$,分别代入核心不等式算出 $f_s$ 的允许范围:
    \begin{itemize}
        \item 当 $m=1$ 时(相当于基带采样):$2f_H \le f_s < \infty$
        \item 当 $m=m_{max}$ 时:通常能得到最低的允许采样频率范围。
    \end{itemize}
    
    \item \textbf{确定最小采样频率}:
    通常题目问的“最小采样频率”位于 $m = m_{max}$ 对应的区间的下限,即:
    \[ f_{s,min} = \frac{2f_H}{m_{max}} \]
    \small{注意:理论上的极限最小值是 $2B$,但只有当 $f_H$ 是 $B$ 的整数倍时,才能精确取到 $2B$。大多数情况下,$f_{s,min}$ 会略大于 $2B$。}
\end{enumerate}
\end{examplebox}