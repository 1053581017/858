\chapter{拉普拉斯变换}

% ==========================================
% 第一部分:核心概念解析
% ==========================================
\section{核心概念:共轭特性与零极点分布}

本节讨论复信号、实信号以及实奇/实偶信号在拉普拉斯域中的对称性规律。

\subsection{1. 基础特性:共轭对称性}
\begin{corebox}{共轭对称公式}
    若 $f(t) \longleftrightarrow F(s)$,则:
    \[ f^*(t) \longleftrightarrow F^*(s^*) \]
    \textbf{直观理解}:时域取共轭,对应频域的“双重共轭”(变量 $s$ 取共轭,函数值也取共轭)。
\end{corebox}

\subsection{2. 实信号的特性 (重点)}
\begin{corebox}{实信号的对称性}
    对于实信号($f(t)$ 为实数),有 $f(t) = f^*(t)$。两边取变换可得:
    \[ F(s) = F^*(s^*) \]
\end{corebox}

\begin{examplebox}{几何意义:零极点分布}
    如果 $s_0$ 是 $F(s)$ 的一个零点(或极点),那么 $s_0^*$ 也必然是 $F(s)$ 的一个零点(或极点)。
    \begin{itemize}
        \item $s_0$ 和 $s_0^*$ 关于\textbf{实轴(横轴)}对称。
        \item \textbf{结论}:实信号的零极点图总是关于实轴对称的。如果有一个复数极点,必然有一个共轭复数极点成对出现。
    \end{itemize}
\end{examplebox}

\subsection{3. 实奇/实偶信号的特性}
如果信号不仅是实信号,还具有奇偶对称性,频域的对称性会更强。

\begin{table}[h!]
    \centering
    \renewcommand{\arraystretch}{1.5}
    \begin{tabularx}{\textwidth}{l|L|L}
        \toprule
        \textbf{时域信号类型} & \textbf{频域 $F(s)$ 特性} & \textbf{零极点分布几何特征} \\
        \midrule
        \textbf{实偶信号} & 偶函数 $F(s) = F(-s)$ & \textbf{四象限对称}:同时关于实轴和虚轴对称。 \\
        \textbf{实奇信号} & 奇函数 $F(s) = -F(-s)$ & \textbf{四象限对称}:零极点位置不受符号影响,同样关于实轴和虚轴对称。 \\
        \bottomrule
    \end{tabularx}
\end{table}
\newpage
% ==========================================
% 第二部分:拉普拉斯变换性质表
% ==========================================
\section{拉普拉斯变换主要性质}

以下性质基于单边拉普拉斯变换整理。

\begin{table}[h!]
    \centering
    \small
    \renewcommand{\arraystretch}{1.8}
    \begin{tabularx}{\textwidth}{@{} l L L l @{}}
        \toprule
        \textbf{性质} & \textbf{连续时间信号} $f(t)$ & \textbf{拉普拉斯变换} $F(s)$ & \textbf{收敛域 (ROC)} \\
        \midrule
        线性 & $a_1 f_1(t) + a_2 f_2(t)$ & $a_1 F_1(s) + a_2 F_2(s)$ & $R_c \supseteq R_1 \cap R_2$ \\
        时移特性 & $f(t-t_0)$ & $e^{-s t_0} F(s)$ & $R_c = R$ \\
        s域移位 & $e^{s_0 t} f(t)$ & $F(s-s_0)$ & $R_c = R + \sigma_0$ \\
        尺度变换 & $f(at)$ & $\frac{1}{|a|} F\left(\frac{s}{a}\right)$ & $R_c = aR$ \\
        共轭特性 & $f^*(t)$ & $F^*(s^*)$ & $R_c = R$ \\
        时域卷积 & $f_1(t) * f_2(t)$ & $F_1(s) F_2(s)$ & $R_c \supseteq R_1 \cap R_2$ \\
        时域微分 & $\frac{df(t)}{dt}$ & $sF(s) - f(0^-)$ & $R_c \supseteq R$ \\
        时域积分 & $\int_{-\infty}^t f(\tau) d\tau$ & $\frac{1}{s} F(s) + \frac{1}{s}\int_{-\infty}^{0^-}f(\tau)d\tau$ & $R_c \supseteq R \cap \{\sigma > 0\}$ \\
        s域微分 & $-t f(t)$ & $\frac{dF(s)}{ds}$ & $R_c = R$ \\
        \midrule
        初值定理 & \multicolumn{3}{l}{$f(0^+) = \lim_{s \to \infty} sF(s)$ \quad (前提:$f(t)$ 在 0 时刻无冲激及其导数)} \\
        终值定理 & \multicolumn{3}{l}{$\lim_{t \to \infty} f(t) = \lim_{s \to 0} sF(s)$ \quad (前提:$sF(s)$ 的极点全在左半平面)} \\
        \bottomrule
    \end{tabularx}
\end{table}
\newpage
% ==========================================
% 第三部分:常用拉普拉斯变换对列表
% ==========================================
\section{常用拉普拉斯变换对列表}

熟记下表中的变换对及收敛域。

\begin{alertbox}{⚠️ 记忆警示}
    记忆时,\textbf{必须配合 ROC 一同记忆}。一定要养成写 ROC 的好习惯,不写 ROC,考试均为零分!
\end{alertbox}

\begin{table}[h!]
    \centering
    \small
    \renewcommand{\arraystretch}{2.0} % 增加行高以容纳分数
    % 修改说明:
    % 第2列和第3列改为 >{\centering\arraybackslash}X,平分剩余空间并居中
    % 第4列改为 c,紧凑适应内容
    \begin{tabularx}{\textwidth}{c | >{\centering\arraybackslash}X | >{\centering\arraybackslash}X | c}
        \toprule
        \textbf{序号} & \textbf{信号} $x(t)$ & \textbf{变换} $X(s)$ & \textbf{收敛域} $\sigma$ \\
        \midrule
        1 & $\delta(t)$ & $1$ & All $s$ \\
        2 & $\delta(t-t_0)$ & $e^{-s t_0}$ & All $s$ \\
        3 & $u(t)$ & $\frac{1}{s}$ & $\sigma > 0$ \\
        4 & $-u(-t)$ & $\frac{1}{s}$ & $\sigma < 0$ \\
        5 & $\frac{1}{n!}t^n u(t)$ & $\frac{1}{s^{n+1}}$ & $\sigma > 0$ \\
        6 & $-\frac{1}{n!}t^n u(-t)$ & $\frac{1}{s^{n+1}}$ & $\sigma < 0$ \\
        7 & $e^{s_0 t} u(t)$ & $\frac{1}{s-s_0}$ & $\sigma > \text{Re}\{s_0\}$ \\
        8 & $-e^{s_0 t} u(-t)$ & $\frac{1}{s-s_0}$ & $\sigma < \text{Re}\{s_0\}$ \\
        9 & $e^{-at} u(t)$ & $\frac{1}{s+a}$ & $\sigma > -a$ \\
        10 & $t e^{-at} u(t)$ & $\frac{1}{(s+a)^2}$ & $\sigma > -a$ \\
        11 & $\frac{1}{n!} t^n e^{-at} u(t)$ & $\frac{1}{(s+a)^{n+1}}$ & $\sigma > -a$ \\
        12 & $e^{-b|t|}, \quad b>0$ & $\frac{-2b}{s^2 - b^2}$ & $-b < \sigma < b$ \\
        13 & $\cos(\omega_0 t) u(t)$ & $\frac{s}{s^2 + \omega_0^2}$ & $\sigma > 0$ \\
        14 & $\sin(\omega_0 t) u(t)$ & $\frac{\omega_0}{s^2 + \omega_0^2}$ & $\sigma > 0$ \\
        15 & $e^{-at} \cos(\omega_0 t) u(t)$ & $\frac{s+a}{(s+a)^2 + \omega_0^2}$ & $\sigma > -a$ \\
        16 & $e^{-at} \sin(\omega_0 t) u(t)$ & $\frac{\omega_0}{(s+a)^2 + \omega_0^2}$ & $\sigma > -a$ \\
        \bottomrule
    \end{tabularx}
\end{table}
核心概念解析:几何法求频率响应我们要分析的是系统函数 $H(s)$ 在虚轴 $s=j\omega$ 上的模值 $|H(j\omega)|$。1. 几何直观对于 $H(s)$ 中的每一个因子:零点 (Zero, $\circ$):代表分子。零点到 $j\omega$ 的距离越短,模值越小(拉低增益)。极点 (Pole, $\times$):代表分母。极点到 $j\omega$ 的距离越短,模值越大(顶高增益,产生峰值)。2. 核心公式$$|H(j\omega)| = K \cdot \frac{\text{所有零点到 } j\omega \text{ 的矢量长度之积}}{\text{所有极点到 } j\omega \text{ 的矢量长度之积}}$$我们在 $j\omega$ 轴上从 $\omega=0$ 走到 $\omega=\infty$,观察这些“矢量长度”是如何变化的,就能画出幅频响应。�� 解题步骤拆解 (针对你上传的例题 7-10)我们依次分析图中的 (a), (b), (c) 三种情况。(a) 低通滤波器 (Low-Pass)系统函数:$H_1(s) = \frac{1}{(s+1)(s+3)}$零极点分布:零点:无(或者说在无穷远处)。极点:实轴上的 $-1$ 和 $-3$。几何分析:$\omega=0$ (直流):参考点在原点。极点矢量长度分别为 1 和 3。模值最大,为 $\frac{1}{1\times3} = \frac{1}{3}$。$\omega \uparrow$ (频率增加):参考点沿虚轴向上移动。极点到 $j\omega$ 的连线(图中 $V_1, V_2$)变长。$\omega \to \infty$:分母中的长度无限变大,导致 $|H(j\omega)| \to 0$。结论:低频大,高频小 $\Rightarrow$ 低通特性。(b) 带通滤波器 (Band-Pass)系统函数:$H_2(s) = \frac{s}{s^2+s+1}$零极点分布:零点:原点 $0$ 有一个零点。极点:共轭复极点 $-\frac{1}{2} \pm j\frac{\sqrt{3}}{2}$。几何分析:$\omega=0$:也就是正好踩在零点上。零点矢量长度为0,分子为0 $\Rightarrow$ 模值为0。$\omega$ 接近极点虚部:当 $\omega$ 走到 $\frac{\sqrt{3}}{2}$ 附近时,距离极点非常近(分母极小),同时距离零点有一定长度(分子不为0)。分母小意味着数值大 $\Rightarrow$ 出现峰值。$\omega \to \infty$:零点矢量长度随 $\omega$ 线性增长(分子),但两个极点矢量长度也随 $\omega$ 增长(分母是 $\omega^2$ 级别)。分母增长得比分子快 $\Rightarrow$ 模值趋向于0。结论:两头低,中间高 $\Rightarrow$ 带通特性。(c) 高通滤波器 (High-Pass)系统函数:$H_3(s) = \frac{s^2}{(s+1)^2}$零极点分布:零点:原点 $0$ 有二阶零点(两个零点重合)。极点:$-1$ 处有二阶极点。几何分析:$\omega=0$:踩在零点上,分子为0 $\Rightarrow$ 模值为0。$\omega \to \infty$:参考点在虚轴无穷远处。此时,零点到参考点的距离($|\omega|$)和极点到参考点的距离($\sqrt{1+\omega^2}$)几乎相等。极限比值:$\frac{\omega^2}{(\sqrt{1+\omega^2})^2} \approx 1$。结论:低频为0,高频通透 $\Rightarrow$ 高通特性。
第一部分:为什么全通系统的零极点关于 $j\omega$ 轴对称?1. 什么是全通系统?全通系统(All-pass System)的定义是:幅度响应恒为常数。也就是说,无论输入信号频率 $\omega$ 是多少,经过系统后,幅度都不变(增益为1),只有相位会发生改变。数学表达:$|H(j\omega)| = C$ (通常归一化为1)。2. 几何直观(最透彻的理解法)我们在复平面上计算 $|H(j\omega)|$ 时,其实是在计算距离的比值:$$|H(j\omega)| = K \cdot \frac{\text{零点到 } j\omega \text{ 的距离}}{\text{极点到 } j\omega \text{ 的距离}}$$想象一下复平面(s平面):$j\omega$ 轴:是我们要考察的“跑道”。极点:通常在左半平面(为了稳定)。假设有一个极点 $p$ 在 $-\sigma$。目标:我们要放置一个零点 $z$,使得你在 $j\omega$ 轴上无论走到哪里,零点到你的距离永远等于极点到你的距离。怎么放?根据平面几何知识,只有当 $j\omega$ 轴是零点和极点连线的垂直平分线时,轴上任意一点到两端的距离才相等。因为 $j\omega$ 轴是垂直的,所以零点必须和极点关于 $j\omega$ 轴镜像对称。如果极点在 $-\sigma$(左),零点就必须在 $+\sigma$(右)。如果极点是复数 $-1+j$,零点就必须是 $+1+j$(实部相反,虚部相同)。3. 数学验证设单极点全通函数:$$H(s) = \frac{s - \sigma}{s + \sigma} \quad (\text{零点在 } +\sigma, \text{ 极点在 } -\sigma)$$令 $s = j\omega$ 求模:$$|H(j\omega)| = \left| \frac{j\omega - \sigma}{j\omega + \sigma} \right| = \frac{\sqrt{\omega^2 + (-\sigma)^2}}{\sqrt{\omega^2 + \sigma^2}} = 1$$结论:正是因为零极点关于虚轴对称,分子分母的模长才始终相等,从而实现“全通”。第二部分:为什么最小相位系统的零极点都在 $j\omega$ 轴左侧?1. 极点为什么在左侧?这比较简单。对于一个因果(Causal)的线性系统,要保证它是稳定(Stable)的,其极点必须全部位于 s 平面的左半平面(LHP)。如果极点在右边,系统响应会包含 $e^{at} (a>0)$,随着时间推移会爆炸式增长,不稳定。2. 零点为什么也要在左侧?这是“最小相位”定义的精髓。我们要从可逆性的角度来理解。定义:最小相位系统是指系统本身是因果稳定的,且其逆系统(Inverse System, $1/H(s)$)也是因果稳定的。推导:原系统 $H(s)$ 的零点,在求逆后变成了逆系统 $1/H(s)$ 的极点。为了让逆系统也稳定,逆系统的极点(也就是原系统的零点)必须也在左半平面。结论:所以最小相位系统的零点必须都在左半平面。3. 物理意义:为什么叫“最小相位”?(能量延迟视角)假设我们要设计一个滤波器,幅度响应已经定死了(比如 $|H(j\omega)|$ 确定了)。我们可以把零点放在左边,也可以把零点镜像到右边(这就变成非最小相位了)。这两种做法幅度是一模一样的(参考第一部分的全通原理),但相位不同。左半平面零点(最小相位):能量集中在时间轴的开始部分。信号响应来得快,延迟小。右半平面零点(非最小相位):能量被推迟了。信号响应会先压低再起来(甚至出现下冲),造成更大的群延迟(Group Delay)。总结对比:对于具有相同幅频响应的一组因果稳定系统:最小相位系统:零点全在左边。它的相位延迟(Phase Lag)最小,群延迟最小,能量最快到达。非最小相位系统:有零点在右边。它等效于一个“最小相位系统”串联了一个“全通系统”。全通系统虽然不改变幅度,但会叠加额外的相位延迟。一句话总结全通系统:为了让分子分母模长相互抵消(幅度恒为1),零极点必须关于虚轴镜像对称。最小相位系统:为了保证逆系统也稳定(或者说为了让信号延迟最小),零点必须和极点一样,乖乖待在左半平面。


 什么是“留数求导法”?留数求导法(Residue Method with Derivatives),也常被称为重极点留数法,是拉普拉斯逆变换中处理**多重极点(Repeated Poles)**部分分式展开的终极工具。为什么需要它?当分母中出现 $(s-p)^n$ 这样的高阶项(比如 $s^3$)时:最高次项(如 $1/s^3$)可以通过简单的“遮盖法”直接求出。低次项(如 $1/s^2$ 和 $1/s$)如果用“遮盖法”直接代入极点,会出现分母为 0 的情况。如果用“通分系数比较法”,计算量大且容易出错。核心原理:泰勒级数展开本质上,我们是将“遮盖”后的剩余函数 $\Phi(s)$ 在极点处进行泰勒级数展开,各项系数正是我们要求的偏分式系数。�� 数学原理推导(直观理解)假设我们要展开的函数为 $F(s)$,它在 $s=p$ 处有一个 $n$ 重极点。形式如下:$$F(s) = \frac{N(s)}{(s-p)^n Q(s)}$$我们要把它展开成:$$F(s) = \frac{K_1}{s-p} + \frac{K_2}{(s-p)^2} + \dots + \frac{K_n}{(s-p)^n} + \dots (\text{其他项})$$第一步:构建核心函数 $\Phi(s)$我们把等式两边同时乘以 $(s-p)^n$,即“遮盖”掉分母中的重极点项:$$\Phi(s) = F(s) \cdot (s-p)^n = \frac{N(s)}{Q(s)}$$等式右边变成:$$\Phi(s) = K_n + K_{n-1}(s-p) + K_{n-2}(s-p)^2 + \dots + K_1(s-p)^{n-1} + (\text{其他项})(s-p)^n$$第二步:像剥洋葱一样求系数求 $K_n$(最高次项系数):直接令 $s=p$。等式右边所有含 $(s-p)$ 的项全部变为 0,只剩下 $K_n$。$$K_n = \Phi(p)$$(这就是普通的遮盖法)求 $K_{n-1}$(次高次项系数):对 $\Phi(s)$ 关于 $s$ 求一阶导数。常数 $K_n$ 消失, $K_{n-1}(s-p)$ 变成 $K_{n-1}$,后面的项仍含有 $(s-p)$。$$\Phi'(s) = 0 + K_{n-1} + 2K_{n-2}(s-p) + \dots$$令 $s=p$,得到:$$K_{n-1} = \Phi'(p)$$求 $K_{n-2}$(再次项系数):对 $\Phi(s)$ 求二阶导数。$$\Phi''(s) = 2K_{n-2} + 3\cdot 2 K_{n-3}(s-p) + \dots$$令 $s=p$,得到 $\Phi''(p) = 2! \cdot K_{n-2}$。所以:$$K_{n-2} = \frac{1}{2!} \Phi''(p)$$