\chapter{拉普拉斯变换}

% ==========================================
% 第一部分:核心概念解析
% ==========================================
\section{核心概念:共轭特性与零极点分布}

本节讨论复信号、实信号以及实奇/实偶信号在拉普拉斯域中的对称性规律。

\subsection{1. 基础特性:共轭对称性}
\begin{corebox}{共轭对称公式}
    若 $f(t) \longleftrightarrow F(s)$,则:
    \[ f^*(t) \longleftrightarrow F^*(s^*) \]
    \textbf{直观理解}:时域取共轭,对应频域的“双重共轭”(变量 $s$ 取共轭,函数值也取共轭)。
\end{corebox}

\subsection{2. 实信号的特性 (重点)}
\begin{corebox}{实信号的对称性}
    对于实信号($f(t)$ 为实数),有 $f(t) = f^*(t)$。两边取变换可得:
    \[ F(s) = F^*(s^*) \]
\end{corebox}

\begin{examplebox}{几何意义:零极点分布}
    如果 $s_0$ 是 $F(s)$ 的一个零点(或极点),那么 $s_0^*$ 也必然是 $F(s)$ 的一个零点(或极点)。
    \begin{itemize}
        \item $s_0$ 和 $s_0^*$ 关于\textbf{实轴(横轴)}对称。
        \item \textbf{结论}:实信号的零极点图总是关于实轴对称的。如果有一个复数极点,必然有一个共轭复数极点成对出现。
    \end{itemize}
\end{examplebox}

\subsection{3. 实奇/实偶信号的特性}
如果信号不仅是实信号,还具有奇偶对称性,频域的对称性会更强。

\begin{table}[h!]
    \centering
    \renewcommand{\arraystretch}{1.5}
    \begin{tabularx}{\textwidth}{l|L|L}
        \toprule
        \textbf{时域信号类型} & \textbf{频域 $F(s)$ 特性} & \textbf{零极点分布几何特征} \\
        \midrule
        \textbf{实偶信号} & 偶函数 $F(s) = F(-s)$ & \textbf{四象限对称}:同时关于实轴和虚轴对称。 \\
        \textbf{实奇信号} & 奇函数 $F(s) = -F(-s)$ & \textbf{四象限对称}:零极点位置不受符号影响,同样关于实轴和虚轴对称。 \\
        \bottomrule
    \end{tabularx}
\end{table}
\newpage

% ==========================================
% 第二部分:拉普拉斯变换性质表
% ==========================================
\section{拉普拉斯变换主要性质}

以下性质基于单边拉普拉斯变换整理。

\begin{table}[h!]
    \centering
    \small
    \renewcommand{\arraystretch}{1.8}
    \begin{tabularx}{\textwidth}{@{} l L L l @{}}
        \toprule
        \textbf{性质} & \textbf{连续时间信号} $f(t)$ & \textbf{拉普拉斯变换} $F(s)$ & \textbf{收敛域 (ROC)} \\
        \midrule
        线性 & $a_1 f_1(t) + a_2 f_2(t)$ & $a_1 F_1(s) + a_2 F_2(s)$ & $R_c \supseteq R_1 \cap R_2$ \\
        时移特性 & $f(t-t_0)$ & $e^{-s t_0} F(s)$ & $R_c = R$ \\
        s域移位 & $e^{s_0 t} f(t)$ & $F(s-s_0)$ & $R_c = R + \sigma_0$ \\
        尺度变换 & $f(at)$ & $\frac{1}{|a|} F\left(\frac{s}{a}\right)$ & $R_c = aR$ \\
        共轭特性 & $f^*(t)$ & $F^*(s^*)$ & $R_c = R$ \\
        时域卷积 & $f_1(t) * f_2(t)$ & $F_1(s) F_2(s)$ & $R_c \supseteq R_1 \cap R_2$ \\
        时域微分 & $\frac{df(t)}{dt}$ & $sF(s) - f(0^-)$ & $R_c \supseteq R$ \\
        时域积分 & $\int_{-\infty}^t f(\tau) d\tau$ & $\frac{1}{s} F(s) + \frac{1}{s}\int_{-\infty}^{0^-}f(\tau)d\tau$ & $R_c \supseteq R \cap \{\sigma > 0\}$ \\
        s域微分 & $-t f(t)$ & $\frac{dF(s)}{ds}$ & $R_c = R$ \\
        \midrule
        初值定理 & \multicolumn{3}{l}{$f(0^+) = \lim_{s \to \infty} sF(s)$ \quad (前提:$f(t)$ 在 0 时刻无冲激及其导数)} \\
        终值定理 & \multicolumn{3}{l}{$\lim_{t \to \infty} f(t) = \lim_{s \to 0} sF(s)$ \quad (前提:$sF(s)$ 的极点全在左半平面)} \\
        \bottomrule
    \end{tabularx}
\end{table}
\newpage

% ==========================================
% 第三部分:常用拉普拉斯变换对列表
% ==========================================
\section{常用拉普拉斯变换对列表}

熟记下表中的变换对及收敛域。

\begin{alertbox}{⚠️ 记忆警示}
    记忆时,\textbf{必须配合 ROC 一同记忆}。一定要养成写 ROC 的好习惯,不写 ROC,考试均为零分!
\end{alertbox}

\begin{table}[h!]
    \centering
    \small
    \renewcommand{\arraystretch}{2.0}
    \begin{tabularx}{\textwidth}{c | >{\centering\arraybackslash}X | >{\centering\arraybackslash}X | c}
        \toprule
        \textbf{序号} & \textbf{信号} $x(t)$ & \textbf{变换} $X(s)$ & \textbf{收敛域} $\sigma$ \\
        \midrule
        1 & $\delta(t)$ & $1$ & All $s$ \\
        2 & $\delta(t-t_0)$ & $e^{-s t_0}$ & All $s$ \\
        3 & $u(t)$ & $\frac{1}{s}$ & $\sigma > 0$ \\
        4 & $-u(-t)$ & $\frac{1}{s}$ & $\sigma < 0$ \\
        5 & $\frac{1}{n!}t^n u(t)$ & $\frac{1}{s^{n+1}}$ & $\sigma > 0$ \\
        6 & $-\frac{1}{n!}t^n u(-t)$ & $\frac{1}{s^{n+1}}$ & $\sigma < 0$ \\
        7 & $e^{s_0 t} u(t)$ & $\frac{1}{s-s_0}$ & $\sigma > \text{Re}\{s_0\}$ \\
        8 & $-e^{s_0 t} u(-t)$ & $\frac{1}{s-s_0}$ & $\sigma < \text{Re}\{s_0\}$ \\
        9 & $e^{-at} u(t)$ & $\frac{1}{s+a}$ & $\sigma > -a$ \\
        10 & $t e^{-at} u(t)$ & $\frac{1}{(s+a)^2}$ & $\sigma > -a$ \\
        11 & $\frac{1}{n!} t^n e^{-at} u(t)$ & $\frac{1}{(s+a)^{n+1}}$ & $\sigma > -a$ \\
        12 & $e^{-b|t|}, \quad b>0$ & $\frac{-2b}{s^2 - b^2}$ & $-b < \sigma < b$ \\
        13 & $\cos(\omega_0 t) u(t)$ & $\frac{s}{s^2 + \omega_0^2}$ & $\sigma > 0$ \\
        14 & $\sin(\omega_0 t) u(t)$ & $\frac{\omega_0}{s^2 + \omega_0^2}$ & $\sigma > 0$ \\
        15 & $e^{-at} \cos(\omega_0 t) u(t)$ & $\frac{s+a}{(s+a)^2 + \omega_0^2}$ & $\sigma > -a$ \\
        16 & $e^{-at} \sin(\omega_0 t) u(t)$ & $\frac{\omega_0}{(s+a)^2 + \omega_0^2}$ & $\sigma > -a$ \\
        \bottomrule
    \end{tabularx}
\end{table}

\newpage

% ==========================================
% 第四部分:核心概念解析
% ==========================================
\section{几何法求频率响应}

我们要分析的是系统函数 $H(s)$ 在虚轴 $s=j\omega$ 上的模值 $|H(j\omega)|$。

\subsection{1. 几何直观}
对于 $H(s)$ 中的每一个因子:
\begin{itemize}
    \item \textbf{零点 (Zero, $\circ$)}:代表分子。零点到 $j\omega$ 的距离越短,模值越小(拉低增益)。
    \item \textbf{极点 (Pole, $\times$)}:代表分母。极点到 $j\omega$ 的距离越短,模值越大(顶高增益,产生峰值)。
\end{itemize}

\subsection{2. 核心公式}
\begin{corebox}{幅频响应几何公式}
    \[ |H(j\omega)| = K \cdot \frac{\text{所有零点到 } j\omega \text{ 的矢量长度之积}}{\text{所有极点到 } j\omega \text{ 的矢量长度之积}} \]
\end{corebox}
我们在 $j\omega$ 轴上从 $\omega=0$ 走到 $\omega=\infty$,观察这些“矢量长度”是如何变化的,就能画出幅频响应。

\subsection{3. 解题步骤拆解 (针对例题 7-10)}
我们依次分析图中的 (a), (b), (c) 三种情况。

\begin{examplebox}{(a) 低通滤波器 (Low-Pass)}
    \textbf{系统函数}:$H_1(s) = \frac{1}{(s+1)(s+3)}$
    \begin{itemize}
        \item \textbf{零极点分布}:零点无(或在无穷远处);极点为实轴上的 $-1$ 和 $-3$。
        \item \textbf{几何分析}:
        \begin{itemize}
            \item $\omega=0$ (直流):极点矢量长度分别为 1 和 3。模值最大,为 $\frac{1}{1\times3} = \frac{1}{3}$。
            \item $\omega \uparrow$ (频率增加):极点到 $j\omega$ 的连线变长。
            \item $\omega \to \infty$:分母中的长度无限变大,导致 $|H(j\omega)| \to 0$。
        \end{itemize}
        \item \textbf{结论}:低频大,高频小 $\Rightarrow$ 低通特性。
    \end{itemize}
\end{examplebox}

\begin{examplebox}{(b) 带通滤波器 (Band-Pass)}
    \textbf{系统函数}:$H_2(s) = \frac{s}{s^2+s+1}$
    \begin{itemize}
        \item \textbf{零极点分布}:原点 $0$ 有一个零点;共轭复极点 $-\frac{1}{2} \pm j\frac{\sqrt{3}}{2}$。
        \item \textbf{几何分析}:
        \begin{itemize}
            \item $\omega=0$:踩在零点上,分子为0 $\Rightarrow$ 模值为0。
            \item $\omega$ 接近极点虚部:当 $\omega \approx \frac{\sqrt{3}}{2}$ 时,距离极点非常近(分母极小),距离零点有一定长度。分母小意味着数值大 $\Rightarrow$ 出现峰值。
            \item $\omega \to \infty$:分子随 $\omega$ 线性增长,分母随 $\omega^2$ 增长。模值趋向于0。
        \end{itemize}
        \item \textbf{结论}:两头低,中间高 $\Rightarrow$ 带通特性。
    \end{itemize}
\end{examplebox}

\begin{examplebox}{(c) 高通滤波器 (High-Pass)}
    \textbf{系统函数}:$H_3(s) = \frac{s^2}{(s+1)^2}$
    \begin{itemize}
        \item \textbf{零极点分布}:原点 $0$ 有二阶零点;$-1$ 处有二阶极点。
        \item \textbf{几何分析}:
        \begin{itemize}
            \item $\omega=0$:踩在零点上,模值为0。
            \item $\omega \to \infty$:零点到参考点距离 ($|\omega|$) 和极点到参考点距离 ($\sqrt{1+\omega^2}$) 几乎相等。极限比值 $\approx 1$。
        \end{itemize}
        \item \textbf{结论}:低频为0,高频通透 $\Rightarrow$ 高通特性。
    \end{itemize}
\end{examplebox}

\section{全通系统与最小相位系统}

\subsection{1. 为什么全通系统的零极点关于 $j\omega$ 轴对称?}

\begin{itemize}
    \item \textbf{定义}:全通系统(All-pass System)的幅度响应恒为常数。即 $|H(j\omega)| = C$。
    \item \textbf{几何直观}:
    我们在复平面上计算 $|H(j\omega)|$ 时,其实是在计算距离的比值:
    \[ |H(j\omega)| = K \cdot \frac{\text{零点到 } j\omega \text{ 的距离}}{\text{极点到 } j\omega \text{ 的距离}} \]
    要使得 $j\omega$ 轴(垂直平分线)上任意一点到零点和极点的距离相等,\textbf{零点必须和极点关于 $j\omega$ 轴镜像对称}。
    \begin{itemize}
        \item 如果极点在 $-\sigma$(左),零点就必须在 $+\sigma$(右)。
        \item 如果极点是复数 $-1+j$,零点就必须是 $+1+j$。
    \end{itemize}
\end{itemize}

\begin{corebox}{数学验证}
    设单极点全通函数 $H(s) = \frac{s - \sigma}{s + \sigma}$ (零点 $+\sigma$,极点 $-\sigma$)。
    令 $s = j\omega$ 求模:
    \[ |H(j\omega)| = \left| \frac{j\omega - \sigma}{j\omega + \sigma} \right| = \frac{\sqrt{\omega^2 + (-\sigma)^2}}{\sqrt{\omega^2 + \sigma^2}} = 1 \]
    \textbf{结论}:正是因为零极点关于虚轴对称,分子分母的模长才始终相等。
\end{corebox}

\subsection{2. 为什么最小相位系统的零极点都在 $j\omega$ 轴左侧?}

\begin{itemize}
    \item \textbf{极点在左侧}:为了保证因果系统的\textbf{稳定性}。
    \item \textbf{零点在左侧}:为了保证\textbf{逆系统}($1/H(s)$)也是稳定因果的。因为原系统的零点会变成逆系统的极点。
    \item \textbf{物理意义(能量延迟视角)}:
    \begin{itemize}
        \item \textbf{最小相位系统}(零点在左):能量集中在时间轴开始部分,相位延迟最小,群延迟最小。
        \item \textbf{非最小相位系统}(有零点在右):等效于“最小相位系统”串联一个“全通系统”。全通系统叠加了额外的相位延迟。
    \end{itemize}
\end{itemize}

\section{留数求导法 (Residue Method with Derivatives)}

\begin{introduction}[方法简介]
    \item \textbf{留数求导法}(也称重极点留数法)是拉普拉斯逆变换中处理\textbf{多重极点}部分分式展开的终极工具。
    \item 当分母出现 $(s-p)^n$ 时,低次项系数无法用简单的遮盖法直接求出,需使用此法。
\end{introduction}

\subsection{数学原理推导}

假设 $F(s) = \frac{N(s)}{(s-p)^n Q(s)}$,展开形式为:
\[ F(s) = \frac{K_1}{s-p} + \frac{K_2}{(s-p)^2} + \dots + \frac{K_n}{(s-p)^n} + \dots \]

\begin{enumerate}
    \item \textbf{第一步:构建核心函数 $\Phi(s)$}
    将两边同时乘以 $(s-p)^n$,即“遮盖”掉重极点项:
    \[ \Phi(s) = F(s) \cdot (s-p)^n = \frac{N(s)}{Q(s)} \]
    此时右边展开式变为多项式形式:
    \[ \Phi(s) = K_n + K_{n-1}(s-p) + K_{n-2}(s-p)^2 + \dots \]

    \item \textbf{第二步:像剥洋葱一样求系数 (泰勒展开)}
    \begin{itemize}
        \item \textbf{求 $K_n$ (最高次项)}:令 $s=p$。
        \[ K_n = \Phi(p) \]
        \item \textbf{求 $K_{n-1}$ (次高次项)}:对 $\Phi(s)$ 求一阶导数后令 $s=p$。
        \[ K_{n-1} = \Phi'(p) \]
        \item \textbf{求 $K_{n-2}$ (再次项)}:对 $\Phi(s)$ 求二阶导数后令 $s=p$,除以 $2!$。
        \[ K_{n-2} = \frac{1}{2!} \Phi''(p) \]
    \end{itemize}
\end{enumerate}

\begin{corebox}{通用公式}
    对于 $(s-p)^n$ 中的倒数第 $m$ 项系数(即对应的 $(s-p)^{n-m+1}$ 项系数):
    \[ K_{n-m} = \frac{1}{m!} \frac{d^m}{ds^m} [\Phi(s)] \Big|_{s=p} \]
\end{corebox}