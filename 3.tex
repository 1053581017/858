\chapter{连续时间信号与系统的傅里叶分析}

% =========================================================
%  NEW INSERTION: 傅里叶级数系数求解方法
% =========================================================

% ==========================================
% 1. 定义法
% ==========================================
\section{定义法 (Direct Definition)}
\begin{corebox}{核心思想}
    \[ \ak = \frac{1}{T} \int_{T} x(t) e^{-jk\wo t} \dd t \]
    \textbf{适用}:简单函数(常数、单项式)。
\end{corebox}

\begin{examplebox}{例 1:周期矩形脉冲}
    $x(t)=1$ for $|t|<T_1$ within $T$.
    \[ \ak = \frac{2\sin(k\wo T_1)}{k\wo T} = \boxed{\frac{\sin(k\wo T_1)}{k\pi}} \]
    \small{注:$k=0$ 时需利用洛必达法则求极限,或直接求面积得 $a_0 = 2T_1/T$ (占空比)。}
\end{examplebox}

% ==========================================
% 2. 微分冲激法
% ==========================================
\section{微分冲激法 (Differentiation)}
\begin{corebox}{原理}
    利用 $x^{(n)}(t) \FT (jk\wo)^n \ak$,将折线求导变冲激。
\end{corebox}

\begin{examplebox}{例 2:周期三角波}
    二次求导得冲激序列 $c_k = \frac{1}{T}(2\cos(k\wo)-2)$。
    \[ \ak = \frac{c_k}{(jk\wo)^2} = \boxed{\frac{2(1-\cos(k\frac{\pi}{2}))}{k^2\pi^2}} \]
\end{examplebox}

% 专门强调 a0 陷阱
\begin{alertbox}{⚠️ 致命陷阱:直流分量丢失}
    \textbf{现象}:微分法求出的 $\ak$ 在 $k=0$ 时分母为 0,公式失效。
    \\ \textbf{物理本质}:求导操作只关注变化率,抹除了信号的“基础海拔”(直流分量)。
    \\ \textbf{对策}:必须回到原图,通过积分计算平均值 $\displaystyle a_0 = \frac{1}{T} \int_T x(t) \dd t$。
\end{alertbox}


% ==========================================
% 3. 关系式法
% ==========================================
\section{关系式法 (Relation to FT)}
\begin{corebox}{公式}
    \[ \ak = \frac{1}{T} X_{0}(jk\wo) \]
    其中 $X_0(\jw)$ 是单周期截断信号的 FT。
    \\ \textbf{物理意义}:周期信号的 $a_k$ 是单周期脉冲频谱的\textbf{采样}。
\end{corebox}

\begin{examplebox}{例 3:周期单边指数信号}
    $x(t) = e^{-t} (0<t<T)$。求得 $X_0(\jw)$ 后代入关系式:
    \[ \boxed{\ak = \frac{1 - e^{-T}}{T(1 + jk\wo)}} \]
\end{examplebox}

% ==========================================
% 4. 等价法
% ==========================================
\section{等价法 (Convolution)}
\begin{corebox}{公式}
    \[ x(t) = \sum x_0(t-kT) \implies \ak = \frac{1}{T} X_0(jk\wo) \]
    本质同方法 3,但物理图景是“卷积”。适用于无限长信号的叠加。
\end{corebox}
% ==========================================
%  NEW SECTION: 周期信号的傅里叶变换
% ==========================================
\subsection{周期信号的傅里叶变换 (FT of Periodic Signals)}
\begin{corebox}{核心公式}
    对于周期为 $T$、基波频率为 $\wo = \frac{2\pi}{T}$ 的周期信号 $x(t)$,其傅里叶变换由一系列冲激函数组成:
    \[ X(j\omega) = \sum_{k=-\infty}^{\infty} 2\pi \ak \delta(\omega - k\wo) \]
    \textbf{物理意义}:
    \begin{itemize}
        \item 周期信号的频谱在频域是\textbf{离散}的。
        \item 能量集中在谐波频率 $k\wo$ 处。
        \item 每个冲激的强度(面积)是其傅里叶级数系数 $\ak$ 的 $2\pi$ 倍。
    \end{itemize}
\end{corebox}

\begin{examplebox}{推导思路}
    利用傅里叶级数展开式 $x(t) = \sum_{k=-\infty}^{\infty} \ak e^{jk\wo t}$,并结合复指数信号的变换对 $e^{j\omega_c t} \FT 2\pi\delta(\omega-\omega_c)$,利用线性性质逐项求和即可得证。
\end{examplebox}
\begin{examplebox}{例 4:重温 4-17(j)}
    $x(t) = \sum e^{-|t-2n|}$ (周期 $T=2$)。
    \[ \ak = \frac{1}{2} X_0(jk\pi) = \boxed{\frac{1}{1 + k^2\pi^2}} \]
\end{examplebox}


% ==========================================
% 5. 考场救命补丁 (核心精华)
% ==========================================
\section{⚡ 考场救命补丁 (Essential Checklists)}
\textit{算完 $a_k$ 后,利用以下直觉快速验算,防止低级错误!}

% --- 建议 1 ---
\subsection*{1. 对称性速查 (Symmetry Check)}

\begin{alertbox}{不计算直接预判虚实性}
    \begin{itemize}
        \item $x(t)$ \textbf{实偶} $\implies \ak$ 必为 \textbf{实偶}。
        \item $x(t)$ \textbf{实奇} $\implies \ak$ 必为 \textbf{纯虚奇}。
        \item $x(t)$ \textbf{半波对称} ($x(t \pm T/2) = -x(t)$) $\implies$ \textbf{偶次谐波 $a_{2k}=0$}。
    \end{itemize}
\end{alertbox}

\begin{bookbox}{}
    \textbf{对应习题:Problem 3.21 ("连连看")}
    \\ 题目给出 6 个波形和 6 组性质,要求\textbf{不计算}直接配对。专门训练“一眼定虚实”的直觉。
\end{bookbox}

% --- 建议 2 ---
\subsection*{2. 收敛性判据 (Decay Rate)}

\begin{alertbox}{看分母阶数知对错}
    \begin{itemize}
        \item 信号有\textbf{跳变} (如方波) $\implies |\ak| \propto \frac{1}{k}$
        \item 信号连续但\textbf{折角} (如三角波) $\implies |\ak| \propto \frac{1}{k^2}$
        \item 信号 $n$ 阶导数才不连续 $\implies |\ak| \propto \frac{1}{k^{n+2}}$
    \end{itemize}
\end{alertbox}

\begin{bookbox}{}
    \textbf{对应习题:Problem 3.16}
    \\ 给出三个不同光滑度的波形(矩形、三角、抛物线),要求连线对应的衰减速率 ($1/k, 1/k^2$)。
\end{bookbox}

% --- 建议 3 ---
\subsection*{3. 功率守恒 (Parseval's Relation)}

\begin{alertbox}{必考公式}
    \[ P_{avg} = \frac{1}{T} \int_T |x(t)|^2 \dd t = \sum_{k=-\infty}^{\infty} |\ak|^2 \]
    \textbf{注意}:这是检查 $a_k$ 幅度是否合理的最终防线。
\end{alertbox}

\newpage

% =========================================================
%  ORIGINAL CONTENT
% =========================================================

\section{帕斯瓦尔定理}
\subsection{能量公式(针对能量信号 / 非周期信号)}

这是帕斯瓦尔定理在\textbf{傅里叶变换(FT)}体系下的形式,通常用于能量有限的非周期信号。

\begin{itemize}
    \item \textbf{物理意义}:信号在时域的总能量等于其在频域的总能量(能量守恒)。
    \item \textbf{数学公式}:
    \begin{itemize}
        \item 使用角频率 $\omega$:
        \begin{equation*}
            \int_{-\infty}^{\infty} |x(t)|^2 dt = \frac{1}{2\pi} \int_{-\infty}^{\infty} |X(\omega)|^2 d\omega
        \end{equation*}
        \item 使用频率 $f$ (Hz):
        \begin{equation*}
            \int_{-\infty}^{\infty} |x(t)|^2 dt = \int_{-\infty}^{\infty} |X(f)|^2 df
        \end{equation*}
    \end{itemize}
    \item \textbf{别名}:此公式有时也被称为 \textbf{瑞利能量定理 (Rayleigh's Energy Theorem)}。
\end{itemize}


\subsection{功率公式(针对功率信号 / 周期信号)}

这是帕斯瓦尔定理在\textbf{傅里叶级数(FS)}体系下的形式,适用于功率有限的周期信号。

\begin{itemize}
    \item \textbf{物理意义}:周期信号在时域的平均功率等于其频域各次谐波分量的功率之和。
    \item \textbf{数学公式}:
    \begin{equation*}
        \frac{1}{T} \int_{T} |x(t)|^2 dt = \sum_{k=-\infty}^{\infty} |a_k|^2
    \end{equation*}
    \item \textbf{参数说明}:
    \begin{itemize}
        \item $T$:信号周期。
        \item $a_k$:傅里叶级数系数(复指数形式)。
        \item 结论:时域的平均功率 = 各次谐波幅度的平方和。
    \end{itemize}
\end{itemize}


\subsection{ 广义帕斯瓦尔公式(乘积形式)}

这是帕斯瓦尔定理的\textbf{一般化形式},描述了两个不同信号在时域的内积与频域内积的关系。

\subsubsection*{A. 公式定义}
\begin{equation*}
    \int_{-\infty}^{+\infty} f(t)g(t)dt = \frac{1}{2\pi}\int_{-\infty}^{+\infty} F(-\omega)G(\omega)d\omega = \frac{1}{2\pi}\int_{-\infty}^{+\infty} F(\omega)G(-\omega)d\omega
\end{equation*}

\subsection*{B. 数学推导逻辑(基于频域卷积定理)}

\begin{enumerate}
    \item \textbf{时频对应关系}:时域的乘积对应频域的卷积(带系数 $\frac{1}{2\pi}$)。
    \begin{equation*}
        f(t)g(t) \longleftrightarrow \frac{1}{2\pi} [F(\omega) * G(\omega)] = \frac{1}{2\pi} \int_{-\infty}^{+\infty} F(u)G(\omega-u) du
    \end{equation*}
    
    \item \textbf{利用变换定义}:傅里叶变换定义为:
    \begin{equation*}
        \mathcal{F}\{x(t)\} = \int_{-\infty}^{+\infty} x(t)e^{-j\omega t} dt
    \end{equation*}
    
    \item \textbf{取特值 $\omega = 0$}:
    \begin{itemize}
        \item 等式左边(时域):令 $\omega=0$,即 $e^{-j0}=1$,变为时域积分 $\int_{-\infty}^{+\infty} f(t)g(t) dt$。
        
\item 等式右边(频域):将 $\omega=0$ 代入卷积公式:
        \begin{equation*}
            = \frac{1}{2\pi} \int_{-\infty}^{+\infty} F(u)G(0-u) du = \frac{1}{2\pi} \int_{-\infty}^{+\infty} F(u)G(-u) du
        \end{equation*}
        \textit{(注:对应图片最右侧等式)}
    \end{itemize}
    
    \item \textbf{变量代换}:令 $v = -u$(则 $du = -dv$),积分限互换抵消负号:
    \begin{equation*}
        = \frac{1}{2\pi} \int_{-\infty}^{+\infty} F(-v)G(v) dv
    \end{equation*}
    \textit{(注:对应图片中间等式)}
\end{enumerate}

\subsection*{C. 与能量公式的退化关系}
\begin{itemize}
    \item \textbf{设定条件}:令 $g(t) = f^*(t)$(即 $f(t)$ 的共轭),且假定 $f(t)$ 为实信号(即 $f^*(t)=f(t)$)。
    \item \textbf{频域性质}:此时 $G(\omega) = F^*(-\omega)$。
    \item \textbf{结果}:公式退化为标准的能量守恒形式:
    \begin{equation*}
        \int_{-\infty}^{+\infty} |f(t)|^2 dt = \frac{1}{2\pi} \int_{-\infty}^{+\infty} |F(\omega)|^2 d\omega
    \end{equation*}
\end{itemize}

\section{常用信号的傅里叶变换及推导笔记}

% --- 原来的 Section 变为 Subsection ---
\subsection{单边指数信号 (Single-sided Exponential)}
\textbf{定义:} $f(t) = e^{-at}u(t), \quad a > 0$

\textbf{推导过程:}
根据傅里叶变换的定义式 $F(\omega) = \int_{-\infty}^{\infty} f(t)e^{-j\omega t} dt$:
\begin{align*}
    F(\omega) &= \int_{0}^{\infty} e^{-at} e^{-j\omega t} dt \\
    &= \int_{0}^{\infty} e^{-(a+j\omega)t} dt \\
    &= \left[ \frac{-1}{a+j\omega} e^{-(a+j\omega)t} \right]_{0}^{\infty}
\end{align*}
由于 $a>0$,当 $t \to \infty$ 时,$e^{-at} \to 0$,故:
\begin{equation}
    F(\omega) = 0 - \left( \frac{-1}{a+j\omega} \cdot 1 \right) = \frac{1}{a+j\omega}
\end{equation}

\textbf{变换对:}
\[ e^{-at}u(t) \longleftrightarrow \frac{1}{a+j\omega} \]

\textbf{频谱特性:}
\begin{itemize}
    \item 幅度谱:$|F(\omega)| = \frac{1}{\sqrt{a^2 + \omega^2}}$
    \item 相位谱:$\varphi(\omega) = -\arctan\left(\frac{\omega}{a}\right)$
\end{itemize}

\hrulefill

\subsection{双边指数信号 (Double-sided Exponential)}
\textbf{定义:} $f(t) = e^{-a|t|}, \quad a > 0$

\textbf{推导过程:}
利用积分区间的可加性,将积分分为 $t<0$ 和 $t>0$ 两部分:
\begin{align*}
    F(\omega) &= \int_{-\infty}^{\infty} e^{-a|t|} e^{-j\omega t} dt \\
    &= \int_{-\infty}^{0} e^{at} e^{-j\omega t} dt + \int_{0}^{\infty} e^{-at} e^{-j\omega t} dt \\
    &= \int_{-\infty}^{0} e^{(a-j\omega)t} dt + \frac{1}{a+j\omega} \quad \text{(引用单边结果)} \\
    &= \left[ \frac{1}{a-j\omega} e^{(a-j\omega)t} \right]_{-\infty}^{0} + \frac{1}{a+j\omega} \\
    &= \left( \frac{1}{a-j\omega} - 0 \right) + \frac{1}{a+j\omega} \\
    &= \frac{1}{a-j\omega} + \frac{1}{a+j\omega} \\
    &= \frac{a+j\omega + a-j\omega}{a^2 + \omega^2}
\end{align*}

\textbf{变换对:}
\begin{equation}
    e^{-a|t|} \longleftrightarrow \frac{2a}{a^2 + \omega^2}
\end{equation}
\textit{注:由于 $f(t)$ 是实偶函数,其频谱 $F(\omega)$ 也是实偶函数。}

\hrulefill

\subsection{矩形脉冲信号 (Rectangular Pulse / Gate Function)}
\textbf{定义:} 
\[ 
g_\tau(t) = \begin{cases} 
1, & |t| < \frac{\tau}{2} \\
0, & |t| > \frac{\tau}{2}
\end{cases} 
\]

\textbf{推导过程:}
\begin{align*}
    G_\tau(\omega) &= \int_{-\tau/2}^{\tau/2} 1 \cdot e^{-j\omega t} dt \\
    &= \left[ \frac{-1}{j\omega} e^{-j\omega t} \right]_{-\tau/2}^{\tau/2} \\
    &= \frac{-1}{j\omega} \left( e^{-j\omega \tau/2} - e^{j\omega \tau/2} \right) \\
    &= \frac{1}{\omega} \cdot \frac{e^{j\omega \tau/2} - e^{-j\omega \tau/2}}{j} \\
    &= \frac{2}{\omega} \sin\left(\frac{\omega \tau}{2}\right)
\end{align*}
引入抽样函数定义 $\text{Sa}(x) = \frac{\sin x}{x}$:
\begin{align*}
    G_\tau(\omega) &= \tau \cdot \frac{\sin(\frac{\omega \tau}{2})}{\frac{\omega \tau}{2}} \\
    &= \tau \text{Sa}\left(\frac{\omega \tau}{2}\right)
\end{align*}

\textbf{变换对:}
\[ g_\tau(t) \longleftrightarrow \tau \text{Sa}\left(\frac{\omega \tau}{2}\right) \]

\hrulefill

\subsection{单位冲激信号 (Unit Impulse)}
\textbf{定义:} $\delta(t)$

\textbf{推导过程:}
利用冲激函数的\textbf{筛选性质}:
\begin{align*}
    F(\omega) &= \int_{-\infty}^{\infty} \delta(t) e^{-j\omega t} dt \\
    &= e^{-j\omega t} \Big|_{t=0} \\
    &= 1
\end{align*}
这表明单位冲激信号包含所有频率分量,且幅度相等。

\textbf{变换对:}
\[ \delta(t) \longleftrightarrow 1 \]

\hrulefill

\subsection{常数信号 (Constant Signal) 与 符号函数}

% --- 原来的 Subsection 变为 Subsubsection ---
\subsubsection{1. 常数信号 $f(t)=1$}
由于 $1$ 不满足绝对可积条件,不能直接积分。利用傅里叶反变换或对偶性推导。
已知 $\delta(t) \leftrightarrow 1$,根据对称性(对偶性):
\[ 1 \longleftrightarrow 2\pi \delta(-\omega) = 2\pi \delta(\omega) \]

\subsubsection{2. 符号函数 $sgn(t)$}
\[ sgn(t) = \begin{cases} 1, & t>0 \\ -1, & t<0 \end{cases} \]
利用极限法 $sgn(t) = \lim_{a \to 0} (e^{-at}u(t) - e^{at}u(-t))$,可得:
\[ sgn(t) \longleftrightarrow \frac{2}{j\omega} \]

\subsubsection{3. 单位阶跃信号 $u(t)$}
利用 $u(t) = \frac{1}{2} + \frac{1}{2}sgn(t)$:
\[ u(t) \longleftrightarrow \pi \delta(\omega) + \frac{1}{j\omega} \]

\hrulefill

\subsection{Sinc 信号 (频域门函数)}
\textbf{定义:} $f(t) = \frac{\sin(\omega_c t)}{\pi t}$

\textbf{推导思路:}
利用对偶性。已知频域为门函数 $G_{2\omega_c}(\omega)$ (宽度为 $2\omega_c$):
\[ F(\omega) = \begin{cases} 1, & |\omega| < \omega_c \\ 0, & \text{其他} \end{cases} \]
其反变换为:
\begin{align*}
    f(t) &= \frac{1}{2\pi} \int_{-\omega_c}^{\omega_c} 1 \cdot e^{j\omega t} d\omega \\
    &= \frac{1}{2\pi} \left[ \frac{1}{jt} e^{j\omega t} \right]_{-\omega_c}^{\omega_c} \\
    &= \frac{1}{\pi t} \cdot \frac{e^{j\omega_c t} - e^{-j\omega_c t}}{2j} \\
    &= \frac{\sin(\omega_c t)}{\pi t}
\end{align*}

\textbf{变换对:}
\[ \frac{\sin(\omega_c t)}{\pi t} \longleftrightarrow \text{Gate}_{2\omega_c}(\omega) \]

% ==========================================
% 1. 基础变形类
% ==========================================
\section{基础变形类 (Basic Deformations)}
\textit{此类性质涉及微积分与函数形状改变,会直接改变 $X(j\omega)$ 的函数形式。}

\begin{table}[h!]
    \centering
    \renewcommand{\arraystretch}{1.6} % 增加行高,让公式更舒展
    \begin{tabularx}{\textwidth}{@{} l l L l @{}}
        \toprule
        \textbf{性质} & \textbf{时域} $x(t)$ & \textbf{频域} $X(j\omega)$ & \textbf{备注} \\
        \midrule
        线性 (Linearity) & $ax_1(t) + bx_2(t)$ & $aX_1(j\omega) + bX_2(j\omega)$ & 叠加原理 \\
        时域微分 & $\frac{d}{dt}x(t)$ & $j\omega X(j\omega)$ & 高频增强 \\
        % 使用 \displaystyle 让积分号显示为正常大小
        时域积分 & $\int_{-\infty}^t x(\tau)d\tau$ & $\displaystyle \frac{1}{j\omega}X(j\omega) + \pi X(0)\delta(\omega)$ & 注意直流分量 \\
        频域微分 & $tx(t)$ & $j\frac{d}{d\omega}X(j\omega)$ & 时域乘 $t$ \\
        频域积分 & $-\frac{1}{jt}x(t) + \pi x(0)\delta(t)$ & $\displaystyle \int_{-\infty}^\omega X(j\eta)d\eta$ & 时域除 $t$ \\
        \bottomrule
    \end{tabularx}
\end{table}

% ==========================================
% 2. 坐标变换类
% ==========================================
\section{坐标变换类 (Coordinate Transformations)}
\textit{此类性质不改变 $X(j\omega)$ 的核心形状,只改变变量 $\omega$ 的位置或比例。}

\begin{table}[h!]
    \centering
    \renewcommand{\arraystretch}{1.5}
    \begin{tabularx}{\textwidth}{@{} l l L l @{}}
        \toprule
        \textbf{性质} & \textbf{时域} $x(t)$ & \textbf{频域} $X(j\omega)$ & \textbf{记忆要点} \\
        \midrule
       
 时移 (Time Shift) & $x(t-t_0)$ & $e^{-j\omega t_0}X(j\omega)$ & 幅度不变,相位线性变化 \\
        频移 (Freq Shift) & $e^{j\omega_0 t}x(t)$ & $X(j(\omega-\omega_0))$ & 调制特性 (Modulation) \\
        尺度变换 (Scaling) & $x(at)$ & $\frac{1}{|a|}X(j\frac{\omega}{a})$ & 时域压缩 $\FT$ 频域扩展 \\
        时间反转 (Reversal) & $x(-t)$ & $X(-j\omega)$ & 若 $x(t)$ 为实数,则共轭 \\
        \bottomrule
    \end{tabularx}
\end{table}

% ==========================================
% 3. 共轭对称性
% ==========================================
\section{共轭对称性 (Conjugate Symmetry)}
\textit{设 $x(t) \FT X(j\omega)$,且 $X(j\omega) = R(\omega) + jI(\omega)$。}

\begin{table}[h!]
    \centering
    
\renewcommand{\arraystretch}{1.4}
    \begin{tabularx}{\textwidth}{@{} l L L @{}}
        \toprule
        \textbf{信号 $x(t)$ 的特性} & \textbf{频谱 $X(j\omega)$ 的特性} & \textbf{数学表达} \\
        \midrule
        一般复信号 & 共轭对称 & $x^*(t) \FT X^*(-j\omega)$ \\
        \textbf{实信号 (Real)} & \textbf{共轭对称} & $X(j\omega) = X^*(-j\omega)$ \\
        & \small{(模偶,相奇,实部偶,虚部奇)} & \\
        \textbf{实偶信号} & \textbf{实偶函数} & $X(j\omega) = R(\omega)$ \small{(纯实数,偶函数)} \\
 
       \textbf{实奇信号} & \textbf{虚奇函数} & $X(j\omega) = jI(\omega)$ \small{(纯虚数,奇函数)} \\
        实信号的偶部 & 对应频谱的\textbf{实部} & $\mathcal{Ev}\{x(t)\} \FT \mathcal{Re}\{X(j\omega)\}$ \\
        实信号的奇部 & 对应频谱的\textbf{虚部} & $\mathcal{Od}\{x(t)\} \FT j\mathcal{Im}\{X(j\omega)\}$ \\
        \bottomrule
    \end{tabularx}
\end{table}

\begin{tcolorbox}[title= 记忆口诀]
实偶对实偶,实奇对虚奇;偶部对实部,奇部对虚部(注意虚部有个 $j$)。
\end{tcolorbox}

% ==========================================
% 4. 交互运算类
% ==========================================
\section{交互运算类 (Interaction)}
\textit{处理两个信号之间的关系,解题核心。}

\begin{itemize}[leftmargin=1cm, itemsep=8pt]
    \item \textbf{时域卷积 (Convolution)}: 解题神器,将卷积变乘法。
    \[ x(t) * h(t) \FT X(j\omega)H(j\omega) \]
    
    \item \textbf{时域相乘 (Modulation)}: 通常用于调制或加窗。
    \[ 
x(t)p(t) \FT \frac{1}{2\pi} [X(j\omega) * P(j\omega)] \]
    
    \item \textbf{对偶性 (Duality)}: 解决 $t$ 域出现 Sinc 函数等难题。
    \[ \boxed{X(t) \FT 2\pi x(-\omega)} \]
    \textcolor{red}{\small (注:此处已修正原笔记中的符号错误)}
\end{itemize}

% ==========================================
% 5. 解题策略
% ==========================================
\section*{ 解题策略:最优计算顺序 (The Algorithm)}
\begin{tcolorbox}[colback=green!5!white, colframe=green!50!black]
当遇到复杂信号求变换时,建议按照以下优先级进行拆解:
\begin{enumerate}
    \item \textbf{交互运算类} (先处理卷积或乘积)
    \item \textbf{基础变形类} (处理微分、积分、乘 $t$)
    \item \textbf{坐标变换类} (最后处理平移、尺度变换)
\end{enumerate}
\end{tcolorbox}

% ==========================================
% 6. 典型推导案例
% ==========================================
\section{典型推导案例:冲激串信号}

\subsection{1. 标准周期冲激串}
设 $p(t) = \sum_{k=-\infty}^{\infty} \delta(t - kT)$,其变换为:
\[ P(j\omega) = \frac{2\pi}{T} \sum_{k=-\infty}^{\infty} \delta\left(\omega - k\frac{2\pi}{T}\right) \]

\subsection{2. 正负交替冲激串 (推导示例)}
\textbf{目标信号:} 周期 $T_0 = 2T$,在一个周期 $[0, 2T)$ 内包含:
\begin{itemize}
    \item $t=0$ 处:$+\delta(t)$
    \item $t=T$ 处:$-\delta(t-T)$
\end{itemize}

\paragraph{Step 1: 求傅里叶级数系数 $a_k$}
\[ a_k = \frac{1}{T_0} \int_{0}^{2T} [\delta(t) - \delta(t-T)] e^{-jk\omega_0 t} dt \]
其中 $\omega_0 = \frac{2\pi}{2T} = \frac{\pi}{T}$。利用冲激函数的筛选性质:
\begin{itemize}
    \item 第一项 $\delta(t)$ 在 $t=0$ 处采样 $\rightarrow e^0 = 1$
    \item 第二项 $-\delta(t-T)$ 在 $t=T$ 处采样 $\rightarrow -e^{-jk\frac{\pi}{T}(T)} = -(-1)^k$
\end{itemize}
得到系数表达式:
\[ a_k = \frac{1}{2T} [1 - (-1)^k] \]
\textbf{分析 $k$ 的奇偶性:}
\begin{itemize}
    \item 当 $k$ 为偶数:$1 - 1 = 0 \implies a_k = 0$
    \item 当 $k$ 为奇数:$1 - (-1) = 
2 \implies a_k = \frac{2}{2T} = \frac{1}{T}$
\end{itemize}

\paragraph{Step 2: 构建傅里叶变换 $Q(j\omega)$}
代入周期信号 FT 通式 $Q(j\omega) = \sum 2\pi a_k \delta(\omega - k\omega_0)$,仅保留奇数项:
\[ Q(j\omega) = \sum_{k=\text{odd}} 2\pi \left( \frac{1}{T} \right) \delta\left(\omega - k\frac{\pi}{T}\right) \]
或者写成更紧凑的形式(令奇数 $k \rightarrow 2m+1$ 或直接保留 $k$):
\[ Q(j\omega) = \frac{2\pi}{T} \sum_{k=-\infty}^{\infty} \delta\left(\omega - (2k+1)\frac{\pi}{T}\right) \]