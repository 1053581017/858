\chapter{线性时不变系统的时域分析}
\section{线性时不变系统的主要特性}
\subsection{记忆性 (Memory)}
系统的记忆特性强调系统的响应是否仅与当前时刻的输入有关。\\
$y(t)=Kx(t)$ 时即为无记忆。

\subsection{因果性 (Causality)}
系统的因果特性强调的是系统的响应是否与未来的输入有关。

\subsection{可逆性 (Invertibility)}
可逆系统由于其输入和响应间存在一一对应关系。

\subsection{稳定性 (Stability)}
有界的输入产生有界的响应。\\
如果 $x(t)$ 前跟一个可以趋于无穷的 $t$(例如 $tx(t)$),则不稳定。

\subsection{时不变性 (Time-Invariance)}
将 $x(t)$ 中的 $t$ 替换为 $t - t_0$,与把方程中所有的 $t$ 都替换为 $t - t_0$ 相等的话,即为时不变。

\subsection{线性 (Linearity)}
若系统满足以下叠加原理,则为线性系统:
\[
T[a \cdot x_1(t) + b \cdot x_2(t)] = a \cdot y_1(t) + b \cdot y_2(t)
\]
即:输入信号的加权和产生的响应,等于各个输入信号单独产生的响应的加权和。

\section{一阶 LTI 系统全响应}
\begin{itemize}
    \item \textbf{全响应 (Total Response)}:一个线性系统的全响应 = 零输入响应 + 零状态响应。
    \item \textbf{零输入响应 ($y_{zi}(t)$)}:仅由系统的\textbf{初始状态}(储能)引起的响应,与外部输入无关。它的形式完全取决于系统的\textbf{特征根}(固有频率)。
    \item \textbf{零状态响应 ($y_{zs}(t)$)}:仅由\textbf{外部输入} $x(t)$ 引起的响应,假设初始状态为零。通常通过卷积 $x(t) * h(t)$ 计算。
\end{itemize}

\section{�� 关键联系:冲激响应 $h(t)$ 与特征方程}

在动手求解之前,理解 $h(t)$ 与微分方程的内在联系能极大地提高解题速度。

对于一阶微分方程 $y'(t) + a y(t) = K x(t)$:

\begin{enumerate}
    \item \textbf{特征方程}:将微分项 $\frac{d}{dt}$ 替换为 $s$,得到 $s + a = 0$。
    \item \textbf{特征根 (Natural Mode)}:解得 $\lambda = -a$。
    \item \textbf{冲激响应 $h(t)$}:
    \begin{itemize}
        \item 物理上,$h(t)$ 是系统受冲激激发后的\textbf{自由振动}。
        \item 数学上,其指数衰减项的系数\textbf{正是特征根}。
        \item 形式为:$h(t) = K e^{\lambda t} u(t) = K e^{-at} u(t)$。
    \end{itemize}
\end{enumerate}

\begin{center}
    \textit{}
\end{center}

\begin{tcolorbox}[colback=yellow!10!white, colframe=orange!85!black, title=⚡️ 秒杀技巧]
只要看到 $h(t) = A \cdot e^{\mathbf{-3}t} u(t)$,立刻可以断定:
\begin{enumerate}
    \item 系统的特征根 $\lambda = \mathbf{-3}$。
    \item 微分方程左边必然是 $y'(t) + \mathbf{3}y(t) = \dots$
    \item 零输入响应 $y_{zi}(t)$ 的形式必然是 $C_{zi} e^{\mathbf{-3}t}$。
\end{enumerate}
\end{tcolorbox}

\section{两种核心解法}

\subsection{��️ 解法一:经典法 (Classical Method)}
\textbf{适用场景}:纯数学计算,关注最终结果而非物理过程,或者求全解时。\\
\textbf{核心公式}:全响应 = 齐次解 $y_h$ + 特解 $y_p$

\begin{enumerate}
    \item \textbf{求齐次解 ($y_h$)}:
    \begin{itemize}
        \item 令输入 $x(t)=0$,列出特征方程(如 $r+a=0$)。
        \item 解出特征根 $\lambda$,写出形式 $y_h(t) = C \cdot e^{\lambda t}$。
    \end{itemize}
    
    \item \textbf{求特解 ($y_p$)}:
    \begin{itemize}
        \item 观察输入 $x(t)$ 的形式(如 $e^{at}$, $\cos \omega t$)。
        \item 设出同形式的待定特解,代入原微分方程求出系数。
    \end{itemize}
    
    \item \textbf{合成并定常数}:
    \begin{itemize}
        \item 写出全解 $y(t) = C \cdot e^{\lambda t} + y_p(t)$。
        \item 代入\textbf{总初值 $y(0^+)$} 解出常数 $C$。
        \item \textit{注意:如果输入包含冲激函数 $\delta(t)$,需先分析 $y(0^-) \to y(0^+)$ 的跳变。}
    \end{itemize}
\end{enumerate}

\subsection{��️ 解法二:双零法 (Zero-Input / Zero-State)}
\textbf{适用场景}:信号与系统分析,需分析“内部储能”与“外部激励”的独立贡献,或输入发生变化时。\\
\textbf{核心公式}:全响应 = 零输入 $y_{zi}$ + 零状态 $y_{zs}$

\begin{enumerate}
    \item \textbf{求零输入响应 ($y_{zi}$)}:
    \begin{itemize}
        \item \textbf{条件}:输入 $x(t)=0$,仅有初值 $y(0^-)$。
        \item \textbf{计算}:它就是齐次解。形式为 $y_{zi}(t) = C_{zi} e^{\lambda t}$。
        \item \textbf{定常数}:直接代入 $y_{zi}(0) = y(0^-)$,解出 $C_{zi}$。
        \item \textit{特性:只要初始状态不变,无论输入怎么变,这部分永远不变。}
    \end{itemize}
    
    \item \textbf{求零状态响应 ($y_{zs}$)}:
    \begin{itemize}
        \item \textbf{条件}:初值 $y(0^-)=0$,仅有输入 $x(t)$。
        \item \textbf{计算途径}:
        \begin{itemize}
            \item \textbf{卷积法}:$y_{zs}(t) = x(t) * h(t)$。
            \item \textbf{微分方程法}:先求特解 $y_p$,然后凑一个齐次解 $C_{zs}e^{\lambda t}$,使得 $y_{zs}(0)=0$。
        \end{itemize}
    \end{itemize}
    
    \item \textbf{合成}:
    \begin{itemize}
        \item $y(t) = y_{zi}(t) + y_{zs}(t)$。
    \end{itemize}
\end{enumerate}
